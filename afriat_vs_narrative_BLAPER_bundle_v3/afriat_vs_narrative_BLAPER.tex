\documentclass[11pt]{article}

% --- Page / typography ---
\usepackage[margin=1in]{geometry}
\usepackage[T1]{fontenc}
\usepackage{lmodern}
\usepackage{microtype}

% --- Math / tables / figures ---
\usepackage{amsmath,amssymb,amsthm}
\usepackage{booktabs}
\usepackage{graphicx}
\usepackage{subcaption}
\usepackage{caption}

% --- Lists / verbatim ---
\usepackage{enumitem}
\usepackage{fancyvrb}

% --- Color / callouts ---
\usepackage{xcolor}
\usepackage[most]{tcolorbox}

% --- Links ---
\usepackage{hyperref}
\hypersetup{
  colorlinks=true,
  linkcolor=blue!60!black,
  urlcolor=blue!60!black,
  citecolor=blue!60!black
}

\graphicspath{{figures/}}

% --- Callout box ---
\newtcolorbox{takeawaybox}{
  colback=blue!3,
  colframe=blue!45!black,
  boxrule=0.6pt,
  arc=2pt,
  left=10pt,
  right=10pt,
  top=7pt,
  bottom=7pt
}

% --- Convenience ---
\newcommand{\X}{\textsc{PapayaTech}}
\newcommand{\Y}{\textsc{AxolotlWorks}}
\newcommand{\Z}{\textsc{QuipuQuantum}}

\title{\textbf{Afriat Sets vs Narrative Rules}\\[0.15em]
\large Counterfactual Prediction in a Portfolio Choice Experiment\\
\large (Paper + Dashboard ``Blaper'')}
\author{}
\date{\vspace{-1.3em}}

\begin{document}
\maketitle

\begin{abstract}
This paper studies counterfactual prediction in a repeated portfolio-allocation environment where each subject allocates a fresh \$100 across three state-contingent assets under changing price vectors.
I compare two paradigms:
(i) Afriat/Varian revealed preference, which delivers a concave-utility \emph{correspondence} of counterfactual optimizers, and
(ii) a narrative/procedural approach that predicts actions via simple, interpretable rules learned from the subject's past decisions.
To evaluate Afriat fairly, I treat it as set-valued and evaluate its counterfactual correspondence as a \emph{set}: for each asset I project the correspondence onto that asset's dollar spend and compute an Afriat band plus a distance-to-band metric.
Across 154 subjects (training on trials 1--25; evaluation at trial 50), these projected Afriat sets are often sharp (median widths around \$4--\$5) but frequently exclude realized behavior (coverage rates around 19--23\% depending on the asset).
On point prediction (train 1--49, predict 50), narrative rule inference materially outperforms Afriat variants in mean $L^1$ error.
\end{abstract}

\vspace{-0.2em}
\begin{takeawaybox}
\textbf{Main synthesis.} \emph{Afriat is a constraint; narratives are a selector.}
Afriat provides disciplined outer structure for counterfactuals (a feasible set).
Narrative procedures provide a selection principle and capture heterogeneity in how subjects behave.
In these data, the main failure mode is often not tie-breaking inside a correct Afriat set, but the Afriat set missing the realized action.
\end{takeawaybox}

\tableofcontents


\section{Reading guide and a translation table}
This paper is intentionally bilingual.

\paragraph{If you come from revealed preference:} start with Section~\ref{sec:tasks} (the prediction games) and Section~\ref{sec:narratives} (narrative rules as model selection). Point prediction at $t=50$ is where Afriat is forced to choose a point (Section~\ref{sec:point}).

\paragraph{If you come from LLMs / AI:} start with Section~\ref{sec:afriat} for what Afriat is doing (and why it produces sets), then Section~\ref{sec:set} for the set-diagnostics dashboard.

\begin{table}[h]\centering
\caption{A Rosetta stone: revealed preference vs prompting language}\label{tab:rosetta}
\begin{tabular}{p{0.28\textwidth}p{0.28\textwidth}p{0.36\textwidth}}
\toprule
Econometrics / RP & Prompting / ML analogue & What we do here \\
\midrule
Observation (trial) $t$ & One row in the prompt ``history'' & \texttt{t=.. prices ... | spend ...} \\
Covariates (prices $p_t$) & Context tokens & price vector shown each period \\
Action (bundle $x_t$) & Model output (JSON) & dollar allocation across three ``stocks'' \\
Counterfactual query price $p_0$ & Test-time query & reveal true prices at $t=50$; ask for allocation \\
Set / partial identification & Non-uniqueness & Afriat gives a correspondence; we score via bands \\
Model selection & Program selection & fixed rule library; personalize via cross-validation \\
\bottomrule
\end{tabular}
\end{table}

\begin{figure}[h]\centering
\includegraphics[width=0.9\textwidth]{task_overview.png}
\caption{Two prediction games used in the BLAPER. Task A: train 1--49, predict 50 (one-step). Task B: train 1--25, generalize to later price regimes (26--50); the set-diagnostics focus on $t=50$.}\label{fig:tasks}
\end{figure}


\section*{Glossary (quick definitions)}
\addcontentsline{toc}{section}{Glossary}
\begin{description}[leftmargin=2.2em,style=nextline]
\item[GARP] Generalized Axiom of Revealed Preference: no cycles in the revealed-preference relation (a nonparametric test of concave utility rationalizability).
\item[Afriat inequalities] Linear inequalities whose feasibility is equivalent to GARP; a feasible solution constructs a concave utility that rationalizes the data.
\item[Demand correspondence] A set of optimal bundles at a given price vector. For piecewise-linear utilities, the correspondence is typically set-valued.
\item[Slack $\delta$ or $\delta_t$] Additive relaxations that ensure feasibility when exact GARP fails. $\delta_t$ is observation-specific.
\item[Afriat band] For a given counterfactual price vector and a given asset, the interval of feasible counterfactual spends implied by Afriat (after projecting the correspondence onto that coordinate).
\item[Coverage] Fraction of subjects whose realized counterfactual spend lies inside the Afriat band.
\item[Distance-to-band] If the truth lies outside the Afriat band, the distance (in dollars) to the nearest band endpoint; 0 if covered.
\item[Narrative rule] A compact subject-specific procedure (e.g., equal-dollar, kNN memory, power-share) inferred from that subject's past choices and then applied to the counterfactual prices.
\item[$L^1$ error] For point prediction, the sum of absolute dollar errors across the three spends at $t=50$.
\end{description}

\section{Data provenance and experimental design}
\subsection{Origin of the experiment}
The data come from the portfolio-choice experiment of Ahn, Choi, Gale, and Kariv (2014).\cite{ahn2014}
Subjects repeatedly choose portfolios over three Arrow securities (three states of nature).
State 2 has known probability $\pi_2=1/3$, while states 1 and 3 have unknown probabilities $\pi_1,\pi_3\ge 0$ with $\pi_1+\pi_3=2/3$.\cite[p.~196]{ahn2014}
In each decision problem the subject chooses a nonnegative portfolio $x=(x_1,x_2,x_3)\ge 0$ on a linear budget set $p\cdot x = 1$ (Ahn et al.'s normalization).\cite[p.~196]{ahn2014}

\subsection{Experimental procedure (what matters here)}
Ahn et al. report:
\begin{itemize}[leftmargin=1.4em]
\item 154 subjects at UC Berkeley; each session consisted of \textbf{50 independent decision problems}.\cite[p.~200]{ahn2014}
\item Each budget set was generated with \textbf{intercepts between 0 and 100 tokens}, and \textbf{at least one intercept greater than 50 tokens}.\cite[p.~200]{ahn2014}
\item Choices were restricted to the budget plane and recorded via a graphical ``point-and-click'' interface.\cite[p.~200]{ahn2014}
\item Payoffs: one decision round was selected at random for payment.\cite[p.~201]{ahn2014}
\end{itemize}

\subsection{Clean file used in this paper}
This paper uses the cleaned dataset \texttt{rationalitydata3goods.csv} with 7700 rows (154 subjects $\times$ 50 trials).
Variables:
\begin{itemize}[leftmargin=1.4em]
\item \texttt{id}: subject identifier; \texttt{obs}: trial index $t\in\{1,\dots,50\}$.
\item \texttt{x,y,z}: chosen quantities $(x_{Xt},x_{Yt},x_{Zt})$.
\item \texttt{xa,ya,za}: budget intercepts (max feasible quantity of each good if all budget is allocated to that good).
\item \texttt{px,py,pz}: prices, with $\texttt{px}\approx 100/\texttt{xa}$ etc.
\end{itemize}
Empirically, the intercept constraints match the protocol: across goods, intercepts lie in $[10,100]$ and each trial has at least one intercept $\ge 50$.
Because choices are recorded on a discrete interface, budget feasibility holds up to small rounding slack (median slack $\approx 0.28$ dollars on a \$100 budget in this cleaned file).

\subsection{Narrative re-framing used for prompts}
For prediction we convert each trial into dollar spends:
\[
s_{it} = p_{it} x_{it},\qquad i\in\{X,Y,Z\},
\]
and define leftover cash $c_t = 100 - (s_{Xt}+s_{Yt}+s_{Zt})$.
We re-label goods as ``stocks'' purely to encourage general reasoning in language models:
\[
X=\X,\quad Y=\Y,\quad Z=\Z.
\]
This is a change of labels/units only.

\section{What is being predicted?}
Each subject faces trials $t=1,\dots,49$ with prices $p_t=(p_{Xt},p_{Yt},p_{Zt})$ and allocates a fresh budget \$100 across the three stocks, yielding dollar spends $(s_{Xt},s_{Yt},s_{Zt})$ and leftover cash.
A new price vector arrives at $t=50$; the goal is counterfactual prediction.

We use two complementary evaluation tracks:
\begin{enumerate}[leftmargin=1.4em]
\item \textbf{Set-based Afriat diagnostics (train 1--25, evaluate at 50).} Fit Afriat with observation-specific slack $\delta_t$ on early history. Compute the counterfactual demand correspondence at $t=50$ and evaluate it as a set (bands, coverage, distance-to-band).
\item \textbf{Point prediction benchmark (train 1--49, predict 50).} Predict the full $t=50$ allocation and evaluate with $L^1$ error (sum of absolute errors across the three spends).
\end{enumerate}

\section{Methods}
\subsection{Mapping dollars to quantities (cash as a fourth good)}
Afriat theory is stated in quantities. We map dollars to quantities and add cash as a fourth good with price 1:
\[
q_{it}=\frac{s_{it}}{p_{it}},\qquad
x_t=(q_{Xt},q_{Yt},q_{Zt},c_t),\qquad
\tilde p_t=(p_{Xt},p_{Yt},p_{Zt},1),
\]
so $\tilde p_t\cdot x_t = 100$.

\subsection{Afriat with observation-specific slack $\delta_t$}
To accommodate violations, we use the relaxation:
\[
u_s \le u_t + \lambda_t\, \tilde p_t\cdot(x_s-x_t) + \delta_t,\qquad \delta_t\ge 0,
\]
estimated by minimizing $\sum_t \delta_t$ subject to these inequalities (an LP).
Given $(u,\lambda,\delta)$, define the concave envelope
\[
U(x)=\min_t\{u_t+\delta_t+\lambda_t\,\tilde p_t\cdot(x-x_t)\}.
\]
At new prices $\tilde p_0$, the counterfactual correspondence is
\[
D(\tilde p_0)=\arg\max_{x\ge 0,\;\tilde p_0\cdot x\le 100} U(x),
\]
which is typically set-valued.

\subsection{Set evaluation via projected bands}
Rather than forcing Afriat into a single point, we evaluate it as a set.
For each stock $i\in\{X,Y,Z\}$, we project $D(\tilde p_{50})$ onto the spend coordinate and compute the band $[\min S_i,\max S_i]$ under near-optimality.
We report:
(i) coverage, (ii) band width, (iii) distance-to-band.


\subsection{Narratives as personalized procedures}\label{sec:narratives}
A narrative model is a compact subject-specific procedure mapping contexts (prices) to actions (allocations).
We restrict to a small interpretable \emph{rule library} and select the best rule per subject via cross-validation on that subject's history.

\paragraph{Rule library (explicit definitions).}
Let prices be $(p_X,p_Y,p_Z)$ and the per-trial budget be 100.
We consider:
\begin{itemize}[leftmargin=1.4em]
\item \textbf{Equal-dollar:} $s_X=s_Y=s_Z=100/3$.
\item \textbf{Equal-share:} equal quantities; in dollars $s_i = 100\cdot p_i/(p_X+p_Y+p_Z)$.
\item \textbf{Power-share:} $s_i = 100\cdot w_i$ with $w_i \propto p_i^{-\alpha}$ (price sensitivity controlled by $\alpha$; $\alpha=0$ yields equal-dollar).
\item \textbf{kNN memory:} predict as an (optionally weighted) average of historical allocations at the $K$ nearest historical price vectors.
\item \textbf{Dominance (all-in):} allocate (nearly) all budget to one stock (e.g., the cheapest).
\end{itemize}
The goal is not to maximize flexible fit; it is to produce a \emph{procedure you can read} and compare to Afriat's disciplined feasibility set.


\section{Results}


\begin{table}[t]\centering
\caption{Afriat as a set at $t=50$ (trained on $t=1..25$): stock-by-stock diagnostics}\label{tab:set}
\begin{tabular}{lrrrrrr}
\toprule
Stock & Coverage & Med. width & Med. dist & Mean dist$\mid$outside & Narrative med. AE & Narrative better$\mid$outside \\
\midrule
PapayaTech & 23.4\% & 4.70 & 11.14 & 22.27 & 11.48 & 61.0\% \\
AxolotlWorks & 20.1\% & 4.42 & 9.53 & 20.60 & 10.57 & 52.0\% \\
QuipuQuantum & 18.8\% & 4.39 & 8.02 & 18.57 & 9.49 & 61.6\% \\
\bottomrule
\end{tabular}
\end{table}

\begin{table}[t]\centering
\caption{Narrative heterogeneity: families inferred from early history ($t=1..25$)}\label{tab:types}
\begin{tabular}{lrr}
\toprule
Narrative family (selected from $t=1..25$) & Count & Share \\
\midrule
Power-share (smooth) & 61 & 39.6\% \\
Case-based (kNN memory) & 41 & 26.6\% \\
Equal-dollar & 34 & 22.1\% \\
Equal-share & 13 & 8.4\% \\
Dominance (all-in) & 5 & 3.2\% \\
\bottomrule
\end{tabular}
\end{table}
\begin{figure}[t]\centering
\includegraphics[width=0.75\textwidth]{narrative_families_bar.png}
\caption{Narrative family counts (selected using early history $t=1..25$).}\label{fig:narr_types}
\end{figure}

\subsection{Afriat set diagnostics by stock}

Figure~\ref{fig:set_papayatech}--\ref{fig:set_quipu} show the distribution of distance-to-band and band width, and how these diagnostics vary with counterfactual price dispersion (ratio of max to min price at $t=50$).

\begin{figure}[t]\centering
\begin{subfigure}[t]{0.48\textwidth}\centering
\includegraphics[width=\textwidth]{papayatech_dist_to_band_hist_t50.png}
\caption{Distance-to-band (t=50)}
\end{subfigure}\hfill
\begin{subfigure}[t]{0.48\textwidth}\centering
\includegraphics[width=\textwidth]{papayatech_band_width_hist_t50.png}
\caption{Band width (t=50)}
\end{subfigure}

\begin{subfigure}[t]{0.48\textwidth}\centering
\includegraphics[width=\textwidth]{papayatech_dist_vs_ratio_t50.png}
\caption{Distance vs price dispersion}
\end{subfigure}\hfill
\begin{subfigure}[t]{0.48\textwidth}\centering
\includegraphics[width=\textwidth]{papayatech_width_vs_ratio_t50.png}
\caption{Width vs price dispersion}
\end{subfigure}
\caption{\X\ (PapayaTech / X): Afriat set diagnostics at $t=50$ (trained on $t=1..25$).}
\label{fig:set_papayatech}
\end{figure}

\begin{figure}[t]\centering
\begin{subfigure}[t]{0.48\textwidth}\centering
\includegraphics[width=\textwidth]{axolotlworks_dist_to_band_hist_t50.png}
\caption{Distance-to-band (t=50)}
\end{subfigure}\hfill
\begin{subfigure}[t]{0.48\textwidth}\centering
\includegraphics[width=\textwidth]{axolotlworks_band_width_hist_t50.png}
\caption{Band width (t=50)}
\end{subfigure}

\begin{subfigure}[t]{0.48\textwidth}\centering
\includegraphics[width=\textwidth]{axolotlworks_dist_vs_ratio_t50.png}
\caption{Distance vs price dispersion}
\end{subfigure}\hfill
\begin{subfigure}[t]{0.48\textwidth}\centering
\includegraphics[width=\textwidth]{axolotlworks_width_vs_ratio_t50.png}
\caption{Width vs price dispersion}
\end{subfigure}
\caption{\Y\ (AxolotlWorks / Y): Afriat set diagnostics at $t=50$ (trained on $t=1..25$).}
\label{fig:set_axolotl}
\end{figure}

\begin{figure}[t]\centering
\begin{subfigure}[t]{0.48\textwidth}\centering
\includegraphics[width=\textwidth]{quipuquantum_dist_to_band_hist_t50.png}
\caption{Distance-to-band (t=50)}
\end{subfigure}\hfill
\begin{subfigure}[t]{0.48\textwidth}\centering
\includegraphics[width=\textwidth]{quipuquantum_band_width_hist_t50.png}
\caption{Band width (t=50)}
\end{subfigure}

\begin{subfigure}[t]{0.48\textwidth}\centering
\includegraphics[width=\textwidth]{quipuquantum_dist_vs_ratio_t50.png}
\caption{Distance vs price dispersion}
\end{subfigure}\hfill
\begin{subfigure}[t]{0.48\textwidth}\centering
\includegraphics[width=\textwidth]{quipuquantum_width_vs_ratio_t50.png}
\caption{Width vs price dispersion}
\end{subfigure}
\caption{\Z\ (QuipuQuantum / Z): Afriat set diagnostics at $t=50$ (trained on $t=1..25$).}
\label{fig:set_quipu}
\end{figure}

\subsection{Point prediction benchmark at $t=50$}


\begin{table}[t]\centering\small
\caption{Point prediction at $t=50$: narrative rules vs Afriat variants (154 subjects)}\label{tab:bench}
\begin{tabular}{p{6.4cm}rrrrrr}
\toprule
Model & Mean $L^1$ & Median $L^1$ & 90th pct $L^1$ & Top-asset acc. & Near all-in pred. & True near all-in \\
\midrule
Rule-based (CV narrative rules) & 31.34 & 18.50 & 83.22 & 55.2\% & 4.5\% & 9.1\% \\
Afriat per-t \ensuremath{\delta_t} + tie-break (closest to anchor) + 0.1-share grid & 47.80 & 31.82 & 115.88 & 53.9\% & 9.7\% & 9.1\% \\
Baseline (mean-share scaling) & 52.51 & 33.82 & 134.34 & 44.2\% & 0.0\% & 9.1\% \\
Afriat per-t \ensuremath{\delta_t} + 0.1-share postprocess & 52.66 & 43.67 & 111.56 & 49.4\% & 13.0\% & 9.1\% \\
Afriat global \ensuremath{\delta} + 0.1-share postprocess & 58.70 & 53.43 & 121.65 & 45.5\% & 18.2\% & 9.1\% \\
Afriat + per-t \ensuremath{\delta_t} (additive) & 58.76 & 50.73 & 127.07 & 52.6\% & 8.4\% & 9.1\% \\
Afriat + global \ensuremath{\delta} (additive) & 64.15 & 58.72 & 134.62 & 44.2\% & 14.3\% & 9.1\% \\
\bottomrule
\end{tabular}
\end{table}

\begin{figure}[t]\centering
\begin{subfigure}[t]{0.48\textwidth}\centering
\includegraphics[width=\textwidth]{t50_L1_hist_baseline_vs_rulebased.png}
\caption{Baseline vs narrative: $L^1$ error}
\end{subfigure}\hfill
\begin{subfigure}[t]{0.48\textwidth}\centering
\includegraphics[width=\textwidth]{t50_L1_hist_afriat_vs_rulebased.png}
\caption{Afriat vs narrative: $L^1$ error}
\end{subfigure}
\caption{Point prediction at $t=50$: distributions of $L^1$ error.}
\label{fig:l1_hists}
\end{figure}

\begin{figure}[t]\centering
\includegraphics[width=0.78\textwidth]{t50_L1_vs_afriat_slack_delta.png}
\caption{$L^1$ error vs fitted Afriat slack at $t=50$: larger fitted slack correlates with worse counterfactual prediction.}
\label{fig:l1_vs_slack}
\end{figure}

\begin{figure}[t]\centering
\includegraphics[width=0.88\textwidth]{t50_L1_vs_ratio_baseline_vs_rulebased.png}
\caption{$L^1$ error vs counterfactual price dispersion: baseline vs narrative.}
\label{fig:l1_vs_ratio}
\end{figure}

\subsection{One-subject deep dive (Subject 920)}
Figure~\ref{fig:subject920} illustrates a representative ``set miss'' case: the Afriat band for \X\ can be narrow yet far from the truth.

\begin{figure}[t]\centering
\begin{subfigure}[t]{0.48\textwidth}\centering
\includegraphics[width=\textwidth]{subject_920_afriat_set_X.png}
\caption{Afriat feasible set for $X$}
\end{subfigure}\hfill
\begin{subfigure}[t]{0.48\textwidth}\centering
\includegraphics[width=\textwidth]{subject_920_X_band_vs_truth.png}
\caption{$X$ band vs truth (t=50)}
\end{subfigure}

\begin{subfigure}[t]{0.48\textwidth}\centering
\includegraphics[width=\textwidth]{subject_920_X_band_width.png}
\caption{$X$ band widths}
\end{subfigure}\hfill
\begin{subfigure}[t]{0.48\textwidth}\centering
\includegraphics[width=\textwidth]{subject_920_L1_timeseries.png}
\caption{Per-period $L^1$ errors}
\end{subfigure}
\caption{Subject 920: Afriat set vs realized behavior (PapayaTech / $X$).}
\label{fig:subject920}
\end{figure}

\section{The dashboard companion (standalone HTML)}
This ``blaper'' is shipped in two synchronized formats:
\begin{itemize}[leftmargin=1.4em]
\item \textbf{PDF (this document):} traditional academic narrative + figures/tables.
\item \textbf{Standalone HTML dashboard:} the same content plus interactive browsing (stock selector, metric hover-glossary, and a subject explorer).
\end{itemize}
Open \texttt{afriat\_vs\_narrative\_BLAPER.html} to use the dashboard view offline.

\section{Conclusion}
Afriat delivers disciplined outer structure for counterfactuals, but counterfactual prediction is inherently a selection problem and, in procedural environments, often calls for procedural abstractions.
Narrative rules provide interpretable, subject-specific procedures that predict well and help diagnose when the Afriat correspondence is missing the realized action.

\appendix
\section{Narrative prompt template (abridged)}
\noindent The evaluation prompts follow a stable template. The full prompts include 49 history lines (t=1..49) and ask for a strict JSON prediction at t=50.

\begin{Verbatim}[fontsize=\small]
Investor Journey — Subject {ID}

Stocks:
  - PapayaTech (ticker: X)
  - AxolotlWorks (ticker: Y)
  - QuipuQuantum (ticker: Z)

Your past 49 trials (t = 1..49):
t=01 prices: X=$..., Y=$..., Z=$... | spend: X=$..., Y=$..., Z=$... | cash=$...
...
t=49 prices: X=$..., Y=$..., Z=$... | spend: X=$..., Y=$..., Z=$... | cash=$...

Now the 50th trial arrives.
t=50 prices: X=$pX50, Y=$pY50, Z=$pZ50

Return ONLY a single JSON object:
{"PapayaTech": 0.00, "AxolotlWorks": 0.00, "QuipuQuantum": 0.00, "Cash": 0.00}
\end{Verbatim}

\begin{thebibliography}{9}

\bibitem{ahn2014}
David Ahn, Syngjoo Choi, Douglas Gale, and Shachar Kariv.
\newblock Estimating ambiguity aversion in a portfolio choice experiment.
\newblock \emph{Quantitative Economics}, 5(2):195--223, 2014.

\bibitem{afriat_1967}
S.~N. Afriat.
\newblock The construction of a utility function from expenditure data.
\newblock \emph{International Economic Review}, 8(1):67--77, 1967.

\bibitem{varian_1982}
H.~R. Varian.
\newblock The nonparametric approach to demand analysis.
\newblock \emph{Econometrica}, 50(4):945--973, 1982.

\end{thebibliography}

\end{document}
